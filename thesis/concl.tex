\chapter{Conclusion}

In this thesis, a system that can recommend articles to the users based upon each user's language level is proposed. A responsive web application with a translation annotation module was created. The application constantly tracks users' interactions with the system through the translation annotation module to determine their language levels. Carver's relative difficulty is adopted in this study which states that one way of measuring relative difficulty is through the percentage of unknown words that a user has towards a specific reading. Based upon the analysis of users' clicking, the system estimates the relative difficulty of each article and labels the relative difficulty of each reading in correspondence to each user's language level. Through multiple experiments using numerous algorithms, it is revealed that the proposed system is able to estimate the relative difficulty for each user towards each reading. The linear regression model was capable of giving the best estimation of Carver's definition of relative difficulty. While the KNNWithZScore is best recommendation algorithm which is evaluated through ranking the articles using the estimated relative difficulty and comparing the estimated ranking with user's own ranking of the difficulty of readings. Due to possible human errors and system malfunction that may have happened during the data collection, a modified relative difficulty called weighted relative difficulty, is proposed in this paper based on the original definition by Carver to reduce the noise in the data. Another experiment was conducted to determine whether the new measurement of the distance between a reading and a user can outperform the original definition by Carver. The result shows that the proposed weighted relative difficulty recommendation model outperforms the previous recommendation model. 

\section{Limitation}

There are many limitations to this study. First, the data sample is quite small and also very possibly biased. Because only friends and families were invited to this study, it is difficult to assure a very balanced dataset. Second, it is a study that is quite hard to replicate the result to ensure its validity, due to the nature that the data sample others collect could be very different from the dataset used in this study. Third, the proposed estimator using collaborative filtering has the same problem that all other recommendation models have such as cold start meaning that it is very difficult for the proposed method to predict the relative difficulty of the readings that no one in the recommendation matrix has read. 

\section{Future work}

Due to the relatively small data sample, it is very hard to state that currently trained prediction model can be used in real life. Enlarge the data sample is definitely necessary for further research on the same topic using similar methods. For a trained prediction model to be used in real life, the problems that exist in the regular recommendation models such as cold start need also be solved. An experiment between learners with different native language using the same system can be conducted to find whether this method is applicable to people with different native language. 
