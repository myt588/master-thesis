語学学習において、言語能力に基づいて読書コンテンツをユーザに推薦するためには、読書の難易度をユーザに対応させて推定することができる必要がありる。難易度は、通常、大量の事前テストによって推定される。しかし、ユーザに多大な労力を強いるため、大規模な規模の読書アプリケーションには適さない。本研究では、事前テストに頼るのではなく、相対的難易度に基づくコンテンツ推薦を行う。相対的難易度はコンテンツ中の未知語の割合であり、ユーザシステムの相互作用を監視することによって実行時に推定することが期待できる。既存研究での相対的難易度はすべての未知語がマークされていると仮定しているが、ユーザの意図しない、もしくは誤りによるクリックミスによって実行時の推定精度が低下する可能性がある。そこで、相人間の誤差の影響を軽減するために、単語の重要度に基づいて重み付けされた加重相対難易度を提案する。また、重み付けされた相対的難易度を推定するために、協調フィルタリングに基づくアルゴリズムを考案する。 提案手法は、328 人のユーザを対象としたケーススタディを通じて評価され、加重相対難易度に基づく推定は、元の相対的難易度に基づく推定よりも高精度であることを示した。
